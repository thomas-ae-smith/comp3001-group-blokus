\section{Description of Prototype Functionality}

\subsection{Blokus}
Blokus is a 2 or 4 player strategic board game, where each player takes turns in placing coloured polyominoes (footnote: A polyomino is a 2D geometric shape formed of one or more equal squares edge-to-edge) on a 20x20 board with the aim of placing all 21 of their pieces. For a 4 player game, each player controls one colour, and for a 2 player game, each player controls two colours. Each player’s first turn must involve placing one of their pieces on one of the four corners of the board. Each subsequent turn must involve placing a piece with only corners touching their existing pieces (so there may be no flat edges of the same coloured shapes touching).  No piece may overlap any other piece. Players may employ tactics to “block” another player from placing their pieces. When no further moves are possible by any player the game is over. A colour loses a point for each square in each unplaced piece. They gain 15 points for placing all their pieces and a further five if the final piece they placed was their monomino (single square).

Client-side validation is used to highlight the area the piece would cover to show whether a position is valid or invalid as appropriate. If the user wishes to rotate or flip the a piece they either hover over it in their piece-tray and use the control halo that appears or once picked up they can use left and right to rotate or ’h’ and ’v’ to flip horizontally and vertically respectively. Once the user has decided where to place it they click once more to place the piece on the board.

\subsection{Authentication}
Users are able to play games without logging in, however, those who register will be able to track their statistics about games they have won and lost. Three methods are available: registering with an account that will be native to the Blokus application, logging in with Facebook, or logging in with Google.

\subsection{Lobby Screen}
From the initial screen that this user is shown - the lobby screen - the user may play a 2 or 4 player game, quick game. Selecting one of these will place the user into a waiting room until there are enough players to start a game, where they will be taken to the game screen. Additionally, the user may select “Private Game”, where they will be given a unique url to share with their friends, so that they can play a game with only people that they know. While waiting for these players, they will be placed in the waiting room.