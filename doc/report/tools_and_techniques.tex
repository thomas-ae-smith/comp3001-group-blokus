\section{List of Tools and Techniques Used}
\subsection{Development Tools}
\begin{itemize}
	\setlength{\itemsep}{1pt}
  	\setlength{\parskip}{0pt}
  	\setlength{\parsep}{0pt}
	\item Various Text Editors with syntax highlighting for general development. No language specific IDEs were used to reduce platform incompatibility issues.
	\begin{itemize}
		\setlength{\itemsep}{1pt}
  		\setlength{\parskip}{0pt}
  		\setlength{\parsep}{0pt}
		\item {Sublime Text 2 (All)}
		\item {Vim (Unix Based Environments)}
	\end{itemize}
	\item Google Chrome Developer Tools and Firebug were used for Javascript debugging.
	\item Mercurial (hosted on BitBucket) was used for collaborative source control.
	\begin{itemize}
		\setlength{\itemsep}{1pt}
  		\setlength{\parskip}{0pt}
  		\setlength{\parsep}{0pt}
		\item {Mercurial (Command Line) (All)}
		\item {TortoiseHG (Windows Environments)}
	\end{itemize}
	\item Mozilla Firefox, Google Chrome, Chromium, IE9 and Safari were used for testing.
	\item mjson.tool was used to check the information in python objects.
	\item gvimdiff, Tortoise Diff, WinMerge, and KDiff3 were used to solve merge conflicts.
	\item cron was used to garbage collect completed or abandoned games and inactive temporary users from the database.
\end{itemize}
	
\subsection{Project Management}
\begin{itemize}
	\item Issue tracking using BitBucket
	\item Google Docs were used for collaborative authoring and document sharing.
	\item Facebook was used for group contact, communication and organisation.
\end{itemize}

\subsection{Techniques}
We used a RAD (Rapid Application Development) approach to development having regular meetings and aiming to have a prototype with more functionality at each one. We also used Facebook and Google Docs to ensure that everyone knew the current state and design of the system.

Temporary data was created on the client side and a DEBUG page to emulate many server functions so that front-end development could continue even if the back-end was not working. In this way the various modules of the system were kept separate as far as possible. Also we used AJAX and JSON based techniques to simplify dynamic web pages and data transfer between the front and back end.