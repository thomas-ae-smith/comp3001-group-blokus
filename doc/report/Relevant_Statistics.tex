% An account of the work to date
\section{Relevant Statistics}
\subsection{External Sources}
\subsubsection{CSS}
\begin{description}
	\item[] Resets all browser layout so all browsers are more likely to be the same.
	\item[] A small script used to place labels inside input text fields.
\end{description}
	
\subsubsection{Python}
\begin{description}
	\item[] Lets Django work with non relational databases like on the GAE.
	\item[] Creates REST interfaces for Django web-services.
	\item[] Allows guest accounts and handles garbage collection for them.
	\item[] Allows authenticating users via Facebook and Google, among others.
\end{description}

\subsubsection{Javascript}
\begin{description}
	\item[jquery] Allows easier manipulation of the DOM.
	\item[underscore] Provides powerful tools to work with advanced data types.
	\item[backbone] Provides simple routing and provides client side representation of models.
	\item[Rapha\"{e}l] Provides a simple and powerful interface for SVG.
\end{description}

\subsection{Relevant Statistics}
\begin{center}
\begin{tabular}{|r|c|c|c|l|}
\hline
\bf{Language} & \bf{Files} & \bf{Comments} & \bf{Code} & \bf{Purpose} \\
\hline 
\bf{Javascript} & 13 & 101 & 1682 & GUI, navigation validation and \newline back-end interaction. \\
\hline
\bf{Python} & 14 & 127 & 777 & Server side scripting. \\
\hline
\bf{CSS} & 1 & 4 & 363 & Client side styling settings. \\
\hline
\bf{HTML} & 6 & 4 & 325 & Templates and static information. \\
\hline
\bf{YAML} & 2 & 0 & 36 & Google App Engin instructions.\\
\hline	
\end{tabular}
\end{center}