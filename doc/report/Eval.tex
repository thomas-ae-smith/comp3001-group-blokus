\section{Evaluation}
When considered as a whole this project has been a success both educationally and productively. It is possible using our system to quickly and easily find and play a game of Blokus. The actual game-play has also been fully implemented including a very responsive and intuitive interface. Further more we have achieved many of our secondary objectives such as a simple and self explanatory lobby allowing 4 player, 2 player and private games. We have also successfully integrated with both Google and Facebook for social authentication but made sure to provide our own authentication system for those who either don’t have or don’t want to share their information from these external sites. We also managed to keep to our rapid-play objective throughout all areas of the game and thus we managed to integrate guest user accounts into our system such that just for a quick game players do not even need to register.

Despite the large range of features implemented other features were considered but not implemented due to the time constraints. We initially considered having the system post to the social networks of our users, device specific controls and, more ambitiously, to create AI players. However it was decided to focus on the aspects of the system more intrinsic to correct operation so these were not implemented.

The Python back-end successfully keeps track of both the game states of those games currently in progress and the user information. The amount of data transmitted is minimised by not creating a game until there are enough players to start the game and only sending the changes in game state rather than the entire game each time. This was an important point to achieve as we wanted the game to feel very responsive. This desire for responsiveness also led to us performing validation on the client side so that the validity of a move can be seen real-time without the need to submit to the server. As no input from the client should be trusted, validation is also implemented server-side, also, after each move Pyhton works out if the game is over and calculates the scores. This is again because anything sent by the client cannot be trusted.

The Javascript side of the system has also been a success. It provides an interactive, intuitive and responsive interface and periodically polls the server to collect any changes to the game state. We also have it performing validation on the fly in order to make it clear to novice users whether a position is valid or not. This is seamlessly integrated into the game flow and does not require the user to do anything extra to gain this ability once more holding to our rapid-play aim.

Overall this project clearly demonstrates our ability to use Javascript to provide responsive and intuitive interfaces with and tie them asynchronously to a back end application written in Python using a Rest API. Both sections of the system demonstrate our ability to model an existing real world system and adapt it for other purposes. Our lobby system shows our ability to handle many discrete requests and appropriately process them. While the system does have areas where it could be extended, it is a fully-functional prototype, and more time and resources the game and surrounding functionality could be further improved. In conclusion we believe that this prototype demonstrates our ability to use the power of scripting languages to combine many components to create a well structured, stable and cohesive system and it has thus achieved its goal.